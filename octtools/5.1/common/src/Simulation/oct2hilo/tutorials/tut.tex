\documentstyle[11pt,titlepage]{article}
\input{/users/staff/paul/tutorials/common/defs.tex}
\input{/users/staff/paul/tutorials/common/header.tex}
\begin{document}
\begin{titlepage}
\begin{center}
\vspace*{2in}
{\LARGE\bf Oct2hilo: Bdnet to HISIM \\[2ex]}
{\Large John D. Sutter \\[1ex]
Massachusetts Microelectronics Center \\[.5ex]
\mmc}
\end{center}

\vspace*{3.2in}
\copyright Copyright 1990 Massachusetts Technology Park Corporation
\end{titlepage}
\tableofcontents

\newpage

\section{Introduction}

  The purpose of this lab is to show how to take an OCT
symbolic description of a circuit and generate a GHDL
description that can be used with System HILO from GenRad.
HISIM, System HILO's timing simulator, can be used
to simulate the circuit using real delays.

  In this lab you will use bdnet from the Octtools to describe
a full adder with standard cells from the MSU 2.2 library.  
You will then use this full adder to build a four bit adder.  
After flattening the hierarchy oct2hilo will be run to convert
the OCT description to GHDL(HILO).  From there you will enter
System HILO, compile the circuit description, and then simulate
it with HISIM.  This tutorial will be broken into two sections:
Octtools and HiLo.


\section{Octtools}

  Figure~\ref{fig:fulladder} shows the full adder you will 
use in this exercise.
The logic gates used directly map into cells from the MSU
library.  The xor gates are XORF201's and the nand gates are
NANF201's.  Listing 1 shows the netlist used to wire these 
cells to make the full adder.  Use your favorite editor to
create the file {\bf fulladder.bdnet} with listing 1.

{\bf Listing 1:}

\begin{verbatim}
MODEL fulladder:symbolic;
TECHNOLOGY scmos;
VIEWTYPE symbolic;
EDITSTYLE symbolic;

OUTPUT sum, cout;
INPUT a, b, cin;
SUPPLY Vdd;
GROUND GND;

INSTANCE "~octtools/lib/technology/scmos/msu/stdcell2_2/xorf201":physical
  ! MSU cell xorf201 xor
  A1: a;
  B1: b;
  O:  a1_xor_b1;
  "Vdd!": Vdd;
  "GND!": GND;

INSTANCE "~octtools/lib/technology/scmos/msu/stdcell2_2/xorf201":physical
  ! MSU cell xorf201 xor
  A1: a1_xor_b1;
  B1: cin;
  O:  sum;
  "Vdd!": Vdd;
  "GND!": GND;

INSTANCE "~octtools/lib/technology/scmos/msu/stdcell2_2/nanf201":physical
  ! MSU cell nanf201 nand
  A1: a;
  B1: b;
  O:  a1_nand_b1;
  "Vdd!": Vdd;
  "GND!": GND;

INSTANCE "~octtools/lib/technology/scmos/msu/stdcell2_2/nanf201":physical
  ! MSU cell nanf201 nand
  A1: a1_xor_b1;
  B1: cin;
  O:  a1_xor_b1_nand_cin2;
  "Vdd!": Vdd;
  "GND!": GND;

INSTANCE "~octtools/lib/technology/scmos/msu/stdcell2_2/nanf201":physical
  ! MSU cell nanf201 nand
  A1: a1_nand_b1;
  B1:  a1_xor_b1_nand_cin2;
  O:  cout;
  "Vdd!": Vdd;
  "GND!": GND;
ENDMODEL;
\end{verbatim}

Run bdnet to create the OCT cell:

{\bf  \% bdnet fulladder.bdnet}

\begin{figure}
\centering
\vspace{1.5in}
\special{psfile=fulladder.ps hoffset=-2.0 voffset=-4.8}
\caption{Full Adder }
\label{fig:fulladder}
\end{figure}

  You now have an OCT symbolic description, {\bf fulladder:symbolic},
of the full adder.  Use bdnet again to wire four full adders 
together to make the four bit adder shown in figure~\ref{fig:4bitadder}.  
Use listing 2 and name the file {\bf 4bitadder.bdnet}.  

{\bf Listing 2:}

\begin{verbatim}
MODEL 4bitadder:symbolic;
TECHNOLOGY scmos;
VIEWTYPE symbolic;
EDITSTYLE symbolic;

OUTPUT sum<3:0>, cout: carry<4> ;
INPUT a<3:0>, b<3:0>, cin: carry<0>;
SUPPLY Vdd;
GROUND GND;

ARRAY %i FROM 0 TO 3 OF
  INSTANCE fulladder:symbolic
    !  gate equivalent of MSU fulladder
    a:    a<%i>;
    b:    b<%i>;
    cin:  carry<%i>;
    sum:  sum<%i>;
    cout: carry<%i+1>;
    Vdd:  Vdd;
    GND:  GND;

ENDMODEL;
\end{verbatim}

Run bdnet on the netlist:

{\bf   \% bdnet 4bitadder.bdnet}

\begin{figure}
\centering
\vspace{3.0in}
\special{psfile=4bitadder.ps hoffset=-1.5 voffset=-3.5}
\caption{Four Bit Adder }
\label{fig:4bitadder}
\end{figure}

  Verify the four bit adder by running musa on the resulting cell, 
{\bf 4bitadder:symbolic} using the control file in listing 3:

{\bf Listing 3:}

\begin{verbatim}
mv a a<3:0>
mv b b<3:0>
mv sum sum<3:0>

ma shsigs
  sh a b cin sum cout
$end

se a 0001
se b 0001
se cin 0
ev
shsigs

se b 1110
ev
shsigs

se cin 1
ev
shsigs

se a 0011
se b 1011
ev
shsigs

se cin 0
ev
shsigs

quit
\end{verbatim}

  Run musa:

{\bf   \% musa -i 4bitadder.ctl 4bitadder:symbolic}

  If things don't add up, check your {\bf .bdnet} files and try again.
You now have a fully functional four bit adder.  To prepare it for 
oct2hilo you must flatten the hierarchy with octflatten.

  Run octflatten:

{\bf   \% octflatten -t LEAF -o 4bitadder:flat 4bitadder:symbolic}

  If you are not familiar with hierarchies you may want to 
extract the netlists from both views using bdnet and compare
them.  You will notice that the netlist for the {\bf symbolic} view
instantiates only the fulladder while the {\bf flat} view instantiates
all the elements from the fulladder:

{\bf   \% bdnet -n 4bitadder:symbolic}\\
{\bf   \% bdnet -n 4bitadder:flat}

  The four bit adder, {\bf 4bitadder:flat}  is now ready to be 
converted to the format used by System HILO.  The conversion tool 
is called oct2hilo.  Oct2hilo takes a flattened description
and converts it GHDL.  Oct2hilo can generate timing parameters
for unit, typical, and worst case delays.  Use typical
delays for now.  For other oct2hilo options see the man page or:

{\bf   \% oct2hilo -usage}

  Running as shown below will generate the GHDL description in 
{\bf 4bitadder.cct}:

{\bf   \% oct2hilo 4bitadder:flat}


\section{System HILO}

  System HILO is a menu driven simulation package.  Before running
a command, arguments and parameters may need to be set.  To run a command
type {\bf run} followed by a carriage return.  There are two ways to set 
a parameter.  The first is by using the cursor keys to get to the
parameter to be changed and then either typing in the value or using
the cursor keys to select a value.  The other way is to set the value
of a parameter from the command line.  For instance, typing 
{\bf name = 4bitadder} would set the name parameter to 4bitadder.

  To start System HiLo type:

{\bf   \% hilo}

  System HILO will put up a title page with a copyright notice.  Hit
the space bar to get to the {\bf CONTENTS} page.  This page lists all other
command pages.  To get to a command page enter its name and then return.
All arguments and parameters will be listed.  

  The GHDL description of the 4bitadder needs to be compiled into your
HILO library.  You probably don't have a HILO library set up yet.  To
create one type:
  
{\bf   \#create}\\
{\bf   \#run}

  This will create the library in your current directory.  You could have
specified another location with the {\bf LDIR} parameter.  Next add
the library of the models for the MSU cells.  This is done with the LIBADD
command page.  Type:

{\bf   \#libadd}

  The directory to add to the path is {\bf \~octtools/lib/oct2hilo/msu\_2.2}.
Unfortunately System HILO doesn't understand the '\~' character so the
entire path will have to be specified.  Determine the path of {\bf ~octtools}
at your site.  As an example we have used {\bf /cad/oct}.  Set the {\bf LIB}
parameter to the MSU HILO library:

{\bf   \#lib=/cad/oct/lib/oct2hilo/msu\_2.2}\\
{\bf   \#run}

  Once a library has been created you probably won't have to do anything
more with it except add new libraries to the search path.  You can list
the libraries in the search path using the {\bf PATHLIST} page.  
You can remove libraries with the {\bf LIBDELETE} page.  
{\bf PATHDELTE} removes all libraries from the search path.

  You should now be ready to compile your circuit.  Go to the {\bf CIRCUIT} 
page.  In the upper right hand corner there is a {\bf NAME} parameter.  
Set this to the name of the circuit and run the compiler:

{\bf   \#circuit}\\
{\bf   \#name=4bitadder}\\
{\bf   \#run}

  If all goes well you will get a message that the circuit compiled.  If 
not it will list the errors.  If the compile fails, make sure you used 
the correct path for the library.  Use the {\bf LIST} command to list the
contents of your library.

{\bf   \#list}\\
{\bf   \#run}

  You have the compiled circuit.  You want to simulate it.  What you
haven't done yet is specify the stimuli to the circuit.  In System
HILO this is done with a {\bf DWL} (Digital Waveform Language) file.  This
file describes what data is placed on the input pins at what time.
{\bf DWL} is quite an extensive language.  Only the basics will be used
for the four bit adder.  

  To create the {\bf DWL} file you can either invoke the editor from System
HILO using vi or invoke your favorite editor using the shell escape
command:

{\bf   \#edit .dwl}\\
    or\\
{\bf   \#!emacs 4bitadder.dwl}

  Note that if you use the {\bf edit} command you don't have to specify 
the full name.  It uses the parameter {\bf name} from the upper right 
hand corner as the first part of the filename and the {\bf .dwl} as the
end.  Create the file using listing 4.

{\bf Listing 4:}

\begin{verbatim}
waveform 4bitadder;
  base bin;
  input TinA[3:0]:=0;
  input TinB[3:0]:=0;
  input Tcin:=0;
  output Tcout;
  output Tsum[3:0];

begin

  TinA:= 0010;
  TinB:= 0010;
  Tcin:=1;
  TinA:= 0011;
  TinB:= 0011;
  TinA:= 0111;
  TinB:= 0111;
  TinA:= 1111;
  TinB:= 1111;
  Tcin:=0;

  TinA:= 1111;
  TinB:= 0000;
  TinB:= 0001;
  Tcin:=1;

  Tcin:=0;
  TinA:= 1111;
  TinB:= 0000;
  Tcin:=1;
  TinB:= 1111;

  Tcin:=0;
  TinA:= 1111
  TinB:= 0111;

  TinA:= 1111
  TinB:= 0010;
end
endwaveform
\end{verbatim}

  To verify the correctness of the {\bf DWL} file use the {\bf WAVEFORM} command
page:

{\bf   \#waveform}\\
{\bf   \#run}

  You actually do not need to run the {\bf WAVEFORM} command on a {\bf DWL}
file.  The {\bf DWL} file is automatically compiled during each simulation.

  There is just one more thing that needs to be done before
simulating the four bit adder.  HISIM, the timing simulator, also needs
a file listing the signals to watch during the simulation.
There three different ways HISIM can display the data from a
simulation run.  The first is the {\bf Visual display} that looks like
a timing table from a data book.  The second is the {\bf Tabular display}
that lists signals by their value in chronological order.  The
third is the {\bf Line display} that gives a verbose listing of signal
values with the simulation time.  These files are referred to by
the suffixes {\bf .vnm}, {\bf .tnm}, and {\bf .lnm} respectively.  Create a
tabular name file using listing 5:

{\bf Listing 5:}

\begin{verbatim}
$time
$space
Tcin
$space
TinA[3:0]
$space
TinB[3:0]
$space
Tcout
$space
Tsum[3:0]
\end{verbatim}

{\bf   \#edit .tnm}\\
    or\\
{\bf   \#!emacs 4bitadder.tnm}

  It is time to simulate.  Go to the {\bf HISIM} command page.  You can
set the three display modes on or off.  The {\bf Tabular display} option
can also direct the output to a file.  Set this option to {\bf BOTH}. 
Run the simulator:

{\bf   \#HISIM}\\
{\bf   \#tab=both}\\
{\bf   \#run}

  If all goes well you will get a screen full of simulation data.
HISIM prompts whether you want to exit or browse the log file.
Hit return to exit.  The best way to examine the data is with 
the editor.  If you choose {\bf BOTH} for the {\bf Tabular display}
parameter HISIM places the data in the file ending in {\bf .tab}:

{\bf   \#edit .tab}\\
    or\\
{\bf   \#!emacs 4bitadder.tab}

  You will notice lines of data blocked together.  These blocks
show the signal propagating through the gates.  By the end of
each block the signals are at the proper levels.  Check some 
of the sums manually to verify the circuit.  To exit System
hilo type:

{\bf   \#exit}
  
  If you have done the {\bf Schematic Capture} tutorial you have probably
noticed a that the implementation of the fulladder is different.  Run
oct2hilo on the flattened 4bit circuit, compile it under System HILO,
and simulate it with HISIM.  You will have to make a {\bf .dwl} 
and a {\bf .tnm} file for it.  Use the files from 4bitadder and 
change the name of the circuit in the {\bf .dwl} file:

{\bf   \#!cp 4bitadder.dwl 4bit.dwl}\\
{\bf   \#e .dwl}\\
{\bf   \#!cp 4bitadder.tnm 4bit.tnm}

  After comparing the simulation with the previous run you will
notice that it takes longer for {\bf cout} to propagate in 4bitadder.


\end{document}

